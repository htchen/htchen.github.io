\documentclass[12pt,twoside]{article}

\usepackage{amsmath}
\usepackage{amsfonts}

\newcommand{\profs}{Hwann-Tzong Chen}
\newcommand{\subj}{09810CS 565300}

\newlength{\toppush}
\setlength{\toppush}{2\headheight}
\addtolength{\toppush}{\headsep}

\newcommand{\htitle}[3]{\noindent\vspace*{-\toppush}\newline\parbox{6.5in}
{\textit{09810CS 565300 -- Statistical Learning Theory}\hfill\newline
National Tsing Hua University \hfill #3\newline
\profs\hfill Assignment #1\vspace*{-.5ex}\newline
\mbox{}\hrulefill\mbox{}}\vspace*{1ex}\mbox{}\newline
\begin{center}{\Large\bf #2}\end{center}}

\newcommand{\assignment}[3]{\thispagestyle{empty}
\markboth{Assignment #1}{Assignment #1}
\pagestyle{myheadings}\htitle{#1}{#2}{#3}}

\setlength{\oddsidemargin}{0pt}
\setlength{\evensidemargin}{0pt}
\setlength{\textwidth}{6.5in}
\setlength{\topmargin}{0in}
\setlength{\textheight}{8.5in}



\begin{document}


\assignment{2}{Assignment 2}{10 October 2009}
\setlength{\parindent}{0pt}

\newcommand{\solution}{
  \medskip
  {\bf Solution:}
}

This assignment is due {\bf Monday October 19} at {\bf 11:59PM}. 

Solutions should be turned in through the assignment FTP site in PDF form.
The name of the PDF file should be ps2\_YourStudentID, e.g., ``ps2\_9762578.pdf''.
Your MATLAB code (with comments) should be included in the PDF file too.
The FTP site is 140.114.71.2 port 1235. You can use the same ID and password of 
the course website to login.
\medskip

\hrulefill

\medskip

\begin{enumerate}


\item {\bf (20 points)} (PRML Exercise 3.2) 

(1) Show that the least-squares solution $\mathbf{y} = \boldsymbol{\Phi}\mathbf{w}_\mathrm{ML} = \boldsymbol{\Phi} \left(\boldsymbol{\Phi}^\mathrm{T} \boldsymbol{\Phi} \right)^{-1} \boldsymbol{\Phi}^\mathrm{T} \mathbf{t}$
corresponds to an {\bf\em orthogonal projection} of the vector $\mathbf{t}$ onto the subspace spanned by the $M$ column vectors $\boldsymbol{\varphi}_j$ of $\boldsymbol{\Phi}$. (See Figure 3.2 of PRML for the illustration and notations.)

(2) Draft an alternative algorithm for approximating $\mathbf{w}_\mathrm{ML}$ without computing $\left(\boldsymbol{\Phi}^\mathrm{T} \boldsymbol{\Phi} \right)^{-1}$ when $\left(\boldsymbol{\Phi}^\mathrm{T} \boldsymbol{\Phi} \right)$ is ill-conditioned.

\medskip
%%%%%%%%%%%%%%%%%%%
\medskip

\item {\bf (20 points)} 

Show that
\begin{equation*}
\begin{split}
& \int (\mathbf{A}\mathbf{x} - \boldsymbol{\mu})^\mathrm{T} \boldsymbol{\Sigma}^{-1} (\mathbf{A}\mathbf{x} - \boldsymbol{\mu}) \,
\mathcal{N}(\mathbf{x};\mathbf{m}, \mathbf{S}) \, d\mathbf{x} \\
& = (\boldsymbol{\mu}-\mathbf{A}\mathbf{m})^\mathrm{T} \boldsymbol{\Sigma}^{-1}(\boldsymbol{\mu}-\mathbf{A}\mathbf{m})
+\mathrm{Tr}\left\{ \mathbf{A}^\mathrm{T} \boldsymbol{\Sigma}^{-1} \mathbf{A} \mathbf{S} \right\}
\end{split}
\end{equation*}




\medskip
%%%%%%%%%%%%%%%%%%%
\medskip

\item {\bf (20 points)} 

Given the likelihood function $p(\mathbf{t}|\mathbf{w}) = \prod_{n=1}^N \mathcal{N}(t_n|\mathbf{w}^\mathrm{T} \boldsymbol{\phi}(\mathbf{x}_n), \beta^{-1})$ and the prior $p(\mathbf{w}) = \mathcal{N}(\mathbf{w}|\mathbf{m}_0, \mathbf{S}_0)$, show that the posterior distribution can be written in the form
\[
p(\mathbf{w}|\mathbf{t}) =  \mathcal{N}(\mathbf{w}|\mathbf{m}_N, \mathbf{S}_N) \,,
\]
\noindent where 
\begin{equation*}
\begin{split}
\mathbf{m}_N &=  \mathbf{S}_N \left( \mathbf{S}_0^{-1} \mathbf{m}_0 + \beta \boldsymbol{\Phi}^\mathrm{T} \mathbf{t} \right) \,, \\
\mathbf{S}_N^{-1} &=  \mathbf{S}_0^{-1}  + \beta \boldsymbol{\Phi}^\mathrm{T} \boldsymbol{\Phi} \,.
\end{split}
\end{equation*}




\medskip
%%%%%%%%%%%%%%%%%%%
\medskip

\item {\bf (40 points)} Reproduce the color image and the three curves shown in Figure 3.10, page 159 of PRML. (MATLAB).

The settings of data and values in your program do not have to be exactly the same as those of Figure 3.10, but the key idea that Figure 3.10 aims to express should be made clear through the illustrations generated by your program.


\end{enumerate}



\end{document}
